%%This is a very basic article template.
%%There is just one section and two subsections.
\documentclass{article}

\begin{document}

A report examining the types of data available to the AstroTrop project, the
potential for data sharing to facilitate new science activities.   

\section{Associated projects}

The process employed for evaluating the available data starts with the list of
project participants and followed links from their web pages to find projects
they were directly or indirectly associated with, and then then looking at the
projects to identifiy what data they collect, what data is available, and where
and how it is stored.
  
\subsection{UN-REDD}
http://www.un-redd.org/AboutUN-REDDProgramme/tabid/102613/Default.aspx

``The UN-REDD Programme is the United Nations collaborative initiative on Reducing
Emissions from Deforestation and forest Degradation (REDD) in developing countries. 

\subsection{REDD++}
http://www.un-redd.org/AboutREDD/tabid/102614/Default.aspx

``Reducing Emissions from Deforestation and Forest Degradation (REDD) is an
effort to create a financial value for the carbon stored in forests, offering
incentives for developing countries to reduce emissions from forested lands	and
invest in low-carbon paths to sustainable development. "REDD+" goes beyond
deforestation and forest degradation, and includes the role of conservation,
sustainable management of forests and enhancement of forest carbon stocks.

The publication section of the REDD++ project website lists a number of useful
standards and guides for data collection and dissemination.
http://www.un-redd.org/PublicationsResources/tabid/587/Default.aspx#technical_work_areas

Question - Does AstroTrop need to take into account, or be compatible with, any
of the standards developed by the REDD and REDD++ projects ?

Question - Should AstroTrop make use of any of the software or
procedures developed by the REDD and REDD++ projects ?

For example : 
\textit{Assessing Forest Governance: A Practical Guide to Data Collection,
Analysis and Use} (http://bit.ly/1qwsAZQ)

The section on data processing mentions the Open Foris toolkit as an example of
toolkits avaiable for collecting and processing survey data.

``For example, the NAFORMA study (see Annex I) used Open Foris, a set of
applications under development by FAO through the FAO-Finland Forestry
Programme.

\subsection{NAFORMA}
 
The \textit{National Forestry Resources Monitoring and Assessment (NAFORMA)}
project in Tanzania is also used as a case study to illustrate how to conduct a
forest governance assessment as part of a large-scale data collection process
for forest monitoring and assessment.

\subsection{NAFORMA case study}
``NAFORMA is a large-scale, field-based study of Tanzania’s forest re-sources as
well as their uses and management. It is the first ground-based inventory of
biophysical and socioeconomic data that covers the entirety of mainland
Tanzania. NAFORMA is designed to be a multi-source forest inventory, allowing
for combining of biophysical field data with remote sensing imagery to produce
accurate data for small areas.

\subsection{NAFORMA - OpenForis}
``This assessment has piloted the FAO-led Open Foris Initiative’s open-source
soft-ware tools (http://www.fao.org/forestry/fma/openforis/en/) and has been
planned, funded, and supported by the Tanzanian government, the Finnish
government, and FAO.

\subsection{NAFORMA - Data access}
``The NAFORMA data-sharing guidelines and communication strategy were developed
in 2013 through a process of stakeholder consultations and national endorsement.

\begin{itemize}
  \item Non-sensitive data will be available for free access
  \item Processed data and .pdf versions of maps will be available in a free and
  transparent manner
  \item Raw data will only be shared where written agreements exist between TFS
  and a collaborating institution and only where the collaborationis
  contributing to a more sustainable management of the forest resources.
\end{itemize}

``FAO Finland is supporting the development of a self-service web platform where
the public can access and query NAFORMA data and results in Open Foris Calc.

\subsection{FAO}
http://www.fao.org/home/en/
``Food and Agriculture Organization of the United Nations
 
\subsection{FAO - Forest monitoring and assessment}
http://www.fao.org/forestry/fma/en/
``FAO's programme dedicated to assisting countries in developing national forest
monitoring systems and assessments with the objective of providing reliable
forest resource information for national forest policy development, planning and
sustainable management.

\subsection{FAO - OpenForis}
http://www.fao.org/forestry/fma/openforis/en/ 
``Open Foris is an FAO-led initiative to develop, share and support specialized
software tools required by countries and institutions to implement multi-purpose
forest inventories.

``It is a set of free and open-source software tools that facilitates flexible
and efficient data collection, analysis and reporting

\subsection{OpenForis}
http://www.openforis.org/home.html
The OpenForis software suite includes 

\subsection{OpenForis - Geospatial}
``Open Foris Geospatial Toolkit is a collection of command-line utilities for
processing of geographical data. It aims to simplify the complex process of
transforming raw satellite imagery for automatic image processing to produce
valuable information. It is particularly useful for processing big amounts of
raster data, and provides a wide range of functionalities including image
manipulation, statistics, segmentation and classification.

\subsection{OpenForis - Collect}
``Collect Earth is a tool that enables data collection through Google Earth.
``In conjunction with Google Earth, Bing Maps and Google Earth Engine, users can
analyze high and very high resolution satellite imagery for a wide variety of
purposes, including :
\begin{itemize}
  \item Land Use, Land Use Change and Forestry (LULUCF) assessments
  \item Monitoring agricultural land and urban areas
  \item Validation of existing maps
  \item Collection of spatially explicit socio-economic data
  \item Quantifying deforestation, reforestation and desertification
  \item Support multi-phase National Forest Inventories
\end{itemize}

\subsection{OpenForis - Calc}
``Open Foris Calc is a robust tool for data analysis and results calculation. The
input data and metadata come from Open Foris Collect and Calc provides a
flexible way to produce aggregated results which can be analyzed and visualized
through the open source software Saiku.

\subsection{Saiku}
http://www.meteorite.bi/
Data analysis and visualization software.






\subsection{Another subtitle}

More plain text.

\end{document}
