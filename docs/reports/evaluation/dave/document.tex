%%This is a very basic article template.
%%There is just one section and two subsections.
\documentclass{article}

%\usepackage{endnotes}

\usepackage{url}	
\usepackage[utf8]{inputenc}
%\usepackage[english]{babel}

\usepackage[style=authoryear,backend=bibtex]{biblatex}

\addbibresource{references}
 
%\setlength{\parindent}{4em}
%\setlength{\parskip}{1em}

\begin{document}

\section{Introduction}

\section{The IVOA Virtual Observatory}

The \citetitle*{ivoa} (\cite{ivoa}) was formed in June 2002 with a mission to
\begin{quote}
"facilitate the international coordination and collaboration
necessary for the development and deployment of the tools, systems and
organizational structures necessary to enable the international utilization of
astronomical archives as an integrated and interoperating virtual observatory."
\end{quote}

The Virtual Observatory (VO) is the realization of the \cite{ivoa} vision of
an integrated and interoperating virtual observatory.
The work of the IVOA focuses on the development of standards, providing a forum
for members to debate and agree the technical standards that are needed to make
the VO possible.

The operational VO itself is comprised of a global shared metadata registry,
the Registry, and a number of individual data discovery and data access services
deployed at each of the participating institutes.
These components work together to present a uniform mechanism for discovering
and accessing data, irrespective of where it is physically located.

The VO architecture and data discovery processes are very similar to the
interconnected metadata collections approach described in
\citetitle{jones-2006} (\cite{jones-2006}).

\begin{quote}
".... a loosely structured collection of project-specific data sets
accompanied by structured metadata about each of the data sets."
\end{quote}

\begin{quote}
"Each of the data sets is stored in a manner that is opaque to the data system in
that the data themselves cannot be directly queried; rather, the structured
metadata describing the data is queried in order to locate data sets of
interest."
\end{quote}

\begin{quote}
"After data sets of interest are located, more detailed information
.... can be extracted from the metadata and used to load, query, and
manipulate individual data sets."
\end{quote}

\subsection{Example use case}

A useful way to illustrate how the data discovery process works in the VO
is to look at an example task such as selecting images covering a particular
region of the sky, in a particular wavelength e.g. infrared, visible light,
radio or xray.

\subsubsection{Service discovery}

The first step of the process is to identify the services that provide access to
the type of data we are looking for by querying the \cite{ivoa.reg}.


The \cite{ivoa.reg} is comprised of a number of small local registry services,
typically hosted at the participating institute level, working in cooperation
with a set of higer level global registry services hosted by a few key institutes
that aggregate the data from the smaller registries to create a global searchable
index of metadata describing all of the services and datasets available in
the VO.

When a new service is deployed, part of the deployment process involves
registering the service with the local registry. The local registry is then
responsible for collecting and storing the metadata that describes both the
service itself and the datasets that it provides access to.

Once the metadata for a serviceor dataset has been registered in a local registry,
it is automatically propagated up to the next level and replicated between the
global registries.

This makes it possible to access the metadata for all of the services and datasets
published in the VO by querying any one of the global registries.

The first step in fulfilling our example use case is to identify services that
contain the type of data we are looking for, in this case images, by querying
the \cite{ivoa.reg} for services that support the \citetitle*{ivoa.sia}
(\cite{ivoa.sia}) capability.

In addition to the technical details of services and their capabilities the
\cite{ivoa.reg} also contains details about the content of datasets, including
details of the wavelength(s) measured, e.g. infrared, visible, radio or xray.

This allows us to refine our query to search for \cite{ivoa.sia} services that
contain images in a specific waveband, e.g. optical, infrared or x-ray.

The \cite{ivoa.reg} query returns a table of data, each row of which contains
information about a \cite{ivoa.sia} service that provides the type of data we are
interested in - images in a particular wavelength.


The VO is itself an evolving system, building on the existing work to add
additional levels of integration as new features are added to the IVOA
specifications.

A recent addition to the list of \cite{ivoa} standards is the
\citetitle*{ivoa.moc} (\cite{ivoa.moc})
which allows \cite{ivoa.reg} services to perform coarse grained region matches.

This will enable us to further refine our \cite{ivoa.reg} query to filter for
\cite{ivoa.sia} services that contained data in a particular
region of the sky.

\subsubsection{Data discovery}

The next stage of the process is to query each of the \cite{ivoa.sia} services
in the list to discover details about the individual images available from that
service.

\noindent
A \cite{ivoa.sia} service can handle queries that specifiy a particular wavelength
and a particular region of the sky :
\begin{itemize}
  \item \texttt{POS}  The positional region (ra, dec)
  \item \texttt{BAND} The energy interval (wavelength)
\end{itemize}

Each \cite{ivoa.sia} service returns a table of data, each row of which
contains metadata about an individual image. The details of the fields in the
image metadata are defined in the
\citetitle*{ivoa.obscore} (\cite{ivoa.obscore})
data model.

This demonstrates a core part of the \cite{ivoa} architecture, interoperable services
based on standard interfaces and data formats.

All of the \cite{ivoa.sia} services will return a standard response, which makes
it much easier to combine them to produce a global list of all the images available
within the whole VO that match our search criteria.

\noindent
The two key components of this are :
\begin{itemize}
  \item A standard interface for the global \cite{ivoa.reg} that uses a standard set of
  attributes to describe datasets and services
  \item A standard interface for local \cite{ivoa.sia} data access services that uses a
  standard set of attributes to describe the available data products
\end{itemize}

The separation between the initial service discovery query at the global
level followed by individual data discovery queries at the local level
is very similar to the stages described in \cite{jones-2006} :
\begin{enumerate}
  \item Querying the metadata to establish the location of suitable data
  \item Querying the individual services to establish what the data is and how to access it
\end{enumerate}

\section{Tropical forest science}

\subsection{Carbon density comparison}

We can compare the VO data discovery process for astronomy data with an example
use case based on a recent study
\citetitle*{mitchard-2014} (\cite{mitchard-2014}),
comparing remote sensing data from satellites with ground plot data collected
in the field.

The study compares two sets of remote sensing data, from
\citetitle*{nasa-jpl-carbon} (\cite{nasa-jpl-carbon})
\citetitle{saatchi-2011} (\cite{saatchi-2011}) [RS1]
and the Woods Hole Research Center
\citetitle{baccini-2012} (\cite{baccini-2012}) [RS2]
with four sets of ground plot data from
\citetitle*{rainfor} (\cite{rainfor})
(\cite{peacock-2007}) (\cite{malhi-2009})
the \citetitle*{atdn} (\cite{atdn}),
the \citetitle*{team} (\cite{team}) network and
the
the \citetitle*{ppbio} (\cite{ppbio}) (\cite{pezzini-2012})

\subsubsection{Remote sensing source data}

It is not know what data discovery and data access methods were used to
identify and access the primary remote sensing source data.

However, there are a number of data discovery tools available that enable
researchers to search for remote sensing data products such as satellite
images and radar scans.

A good examples of this type of tool are the
\cite{usgs-explorer}
and
\cite{usgs-glovis}
tools provided by the
\citetitle*{usgs} (\cite{usgs})
    
\begin{quote}
"The USGS EarthExplorer ... provides users the ability to query, search,
and order satellite images, aerial photographs, and cartographic products from
several sources"
\end{quote}

\begin{quote}
"In addition to data from the Landsat missions and a variety of other data providers,
EE now provides access to MODIS land data products from the NASA Terra and Aqua missions,
and ASTER level-1B data products over the U.S. and Territories from the NASA ASTER mission"
\end{quote}

\begin{quote}
"The USGS Global Visualization Viewer (GloVis) is an online search and order tool for
selected satellite data. The viewer allows access to all available browse images
from the Landsat 7 ETM+, Landsat 4/5 TM, Landsat 1-5 MSS, EO-1 ALI, EO-1 Hyperion,
MRLC, and Tri-Decadal data sets, as well as Aster TIR, Aster VNIR and MODIS browse
images from the DAAC inventory"
\end{quote}

The \cite{usgs} also provides large area composited mosaics generated from
\cite{landsat}
data via the
\cite{weld}
project.

\begin{quote}
"The WELD data products are processed so users do not need to apply the equations,
spectral calibration coefficients, and solar information, needed to convert Landsat
digital numbers to reflectance and brightness temperature.
They are defined in the same coordinate system and align precisely, making them simple
to use for multi-temporal applications.
The products provide consistent data that can be used to derive higher-level land
cover as well as geo-physical and biophysical products for assessment of surface
dynamics and to study Earth system functioning"
\end{quote}

The \cite{usgs} also maintains a
\citetitle*{usgs-lta} (\cite{usgs-lta})
of historical remote sensing data.

\begin{quote}
"The U.S. Geological Survey's (USGS) Long Term Archive (LTA) at the National Center
for Earth Resource Observations and Science (EROS) in Sioux Falls, SD is one of
the largest civilian remote sensing data archives"
\end{quote}

\begin{quote}
"Time series images are a valuable resource for scientists, disaster managers,
engineers, educators, and the general public. USGS EROS has archived, managed,
and preserved land remote sensing data for more than 35 years and is a leader
in preserving land remote sensing imagery"
\end{quote}

However, all of these interfaces are based around human interaction.
As far as we know, at the time of writing, there are no machine readable
data discovery services for remote sensing source data.

\subsubsection{Carbon density maps}

A detailed description of the [RS1] dataset produced by \citetitle*{nasa-jpl-carbon}
is available in the authors paper (\cite{saatchi-2011}).

The paper, along with the additional supporting information available on the
\cite{pnas} website, describes the main upstream data sources and the methods applied.

However, details of the data sources, the instruments, target areas and date ranges
the data covers are not available in a machine readable format.

\begin{quote}
"Ground data used to train the biomass prediction model were derived from
various sources including published literature and national forest inventories
collected by the authors and their colleagues."
\end{quote}

The carbon density dataset itself is available as \cite{format-geotiff} files,
with associated \cite{format-world} metadata, for download from the
\cite{nasa-jpl-carbon-ftp} site.

A detailed description of the RS2 dataset produced by the \citetitle*{whrc}
is available in the authors paper (\cite{baccini-2012}).

The paper, along with the additional supporting information available from
the \cite{journal-nature} website,
describes the upstream data sources and the methods applied.
However, details of the data sources, the instruments, target areas and date ranges
the data covers are not available in a machine readable format.

The carbon density dataset itself is available by request from the 
\cite{whrc-data} website.
Access to the data requires filling in a simple web form declaring
who you are and what you want to use the data for.
On submitting the web-form, an automated email reply is generated
containing a URL to a \cite{format-zip}
file on the WHRC website.

The \cite{format-zip} file contains the data as \cite{format-geotiff}
files, with associated \cite{format-world} metadata.

\subsubsection{Ground plot data}

The four sets of ground plot data from
\cite{rainfor}, \cite{atdn}, \cite{team}
and \cite{ppbio}
were combined together in the 
\cite{forest-plots}
database.

Details fo the design and capabilities of the \cite{forest-plots}
system is presented in \citetitle*{gonzalez-2011} (\cite{gonzalez-2011}).

\begin{quote}
"The ForestPlots.net web application was designed primarily
as a repository for long-term intact tropical forest inventory
plots, where trees within an area are individually identified,
measured and tracked through time"
\end{quote}

Of the three sets of ground plot data, the data from \cite{rainfor} and
\cite{atdn} were already available in the \cite{forest-plots} database.

The plot data from the \cite{team} and \cite{ppbio} projects were
downloaded and imported into the \cite{forest-plots} database manually.

The principal \cite{term-agb}
data was calculated using one of the built-in
\cite{comp-lang-sql}
queries provided by the
\cite{forest-plots}
database system which implements the tropical forest model described in
\citetitle*{chave-2005} (\cite{chave-2005})

The results of the \cite{term-agb} calculation were stored in the \cite{forest-plots}
database as a new dataset available for download from the the \cite{forest-plots} site
and referenced as part of the source material for the paper.





\section{Metadata formats}

Within the set of datasets used by use cases, we can see a variety of
different database systems storing different types of metadata in a varierty of
different structures and formats.

\subsubsection{Global Index of Vegetation-Plot Databases}

The Global Index of Vegetation-Plot Databases (GIVD)
%%endnote{http://www.givd.info/}
is a database of metadata describing databases of vegetation plot data
from around the world.

\noindent
ForestPlots.Net is described in the GIVD database
\texttt{[GIVD:00-00-001]}
%%endnote{http://www.givd.info/ID/00-00-001}
as is the PPBio information system
\texttt{[GIVD:SA-BR-001]}	
%%endnote{http://www.givd.info/ID/SA-BR-001}
and the data from the TEAM network
\texttt{[GIVD:00-00-002]}.
%%endnote{http://www.givd.info/ID/00-00-002}

Dengler et al. describe the GIVD project in a 2011 paper,
"The Global Index of Vegetation-Plot Databases (GIVD): a new resource for vegetation science"
(Dengler et al. 2011)
%%endnote{doi:10.1111/j.1654-1103.2011.01265.x}
%%endnote{http://onlinelibrary.wiley.com/doi/10.1111/j.1654-1103.2011.01265.x/abstract}
and suggest some future applications, including the idea of combining
different types of data from different, distributed, databases.

\begin{quote}
"Our longer-term vision is to develop GIVD in ways similar to Metacat (Jones et
al. 2006), so that, ultimately, users who query GIVD will not only receive
information on which databases contain data suitable for the intended analyses,
but they will also discover other data from distributed databases, with GIVD
acting as the central node."
\end{quote}

\begin{quote}
"By coupling species specific trait characteristics (e.g. mean plant height,
specific leaf area, growth form) found in trait databases, such as LEDA (Kleyer
et al. 2008) or TRY (http://www.trydb.org), with plot-based distribution
information on those species, GIVD could support further refinement of DGVMs."
\end{quote}

Which is similar to the distributed architectrure of data discovery and data
access services used by the virtual observatory.

\subsubsection{METACAT}

Different institutes have different emphasis and different
aproaches to handling the metadata associated with 
 
In a 2012 paper by Flávia Fonseca Pezzini et al. about the PPBio project 
(Pezzini et al. 2012)
"The Brazilian Program for Biodiversity Research (PPBio) Information System"
%%endnote{http://www.biodiversity-plants.de/biodivers_ecol/article_meta.php?DOI=10.7809/b-e.00083}
they describe the role of the data manager and the metadata
collection processes that are in place.

They also describe the transition from data storage in flat files,
which was sufficient for the first five years of the project,
to a new system based on Metacat.

\begin{quote}
To facilitate data searches, all the metadata were converted to XML,
and the PPBio has installed a METACAT server to integrate with the
Knowledge Network for Biocomplexity (KNB), a network which aims to
assist ecological and environmental research.
\end{quote}

Metacat
%%endnote{http://knb.ecoinformatics.org/software/metacat}
%%endnote{https://knb.ecoinformatics.org/knb/docs/intro.html}
is an open source data management tool that provides a repository for
managing both data and metadata in one system.

\begin{quote}
Metacat is a repository for data and metadata (documentation about data) that
helps scientists find, understand and effectively use data sets they manage or
that have been created by others.
\end{quote}

Metacat is capable of handling a variety of different metadata formats,
including Ecological Metadata Language (EML)
%%endnote{https://knb.ecoinformatics.org/#external//emlparser/docs/index.html}
FGDC Biological Data Profile.
%%endnote{https://www.fgdc.gov/standards/projects/FGDC-standards-projects/metadata/biometadata}

\subsubsection{DataONE}

The Metacat project is itself part of the Data Observation Network for Earth (DataONE)
%%endnote{https://www.dataone.org/}
project, a collaboration spconsored by the U.S. National Science Foundation to build
an infrastructire from distributed webservices that provides open, persistent, robust,
and secure access to Earth observational data.
 
\begin{quote}
The DataONE project is a collaboration among scientists, technologists,
librarians, and social scientists to build a robust, interoperable, and
sustainable system for preserving and accessing Earth observational data at
national and global scales. Supported by the U.S. National Science Foundation,
DataONE partners focus on technological, financial, and organizational
sustainability approaches to building a distributed network of data repositories
that are fully interoperable, even when those repositories use divergent
underlying software and support different data and metadata content standards.
\end{quote}

The DataONE arcitecture is based on a set of top level 
\textit{Coordinating Nodes}
and
\textit{Member Nodes}
located at each participating institute or organisation

\textit{Coordinating Nodes}
provide a replicated catalog of Member Node holdings, enabling
scientists to discover data wherever they reside,
and data repositories to make their data and services available
to the international community.

The individual \textit{Member Nodes}
at each institute enable them to make their data available
to the rest of the DataONE network
via a standard webservice interface.

Again, this two layers of data discovery and data access
is similar the virtual observatory architecture.










 


















%\theendnotes

%\bibliographystyle{plain}
%\bibliography{sample}

\printbibliography

\end{document}
