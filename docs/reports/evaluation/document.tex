%%This is a very basic article template.
%%There is just one section and two subsections.
\documentclass{article}

\usepackage{endnotes}
	
\usepackage[utf8]{inputenc}
\usepackage[english]{babel}
 
\setlength{\parindent}{4em}
\setlength{\parskip}{1em}

\begin{document}

\section{Introduction}

Some text ..

\section{The IVOA}

The International Virtual Observatory Alliance (IVOA) was formed in June 2002
with a mission to "facilitate the international coordination and collaboration
necessary for the development and deployment of the tools, systems and
organizational structures necessary to enable the international utilization of
astronomical archives as an integrated and interoperating virtual observatory."

The work of the IVOA focuses on the development of standards, providing a forum
for members to debate and agrees the technical standards that are needed to make
the VO possible.

\section{The VO}

The Virtual Observatory (VO) is the realization of the IVOA vision of an
\textit{``an integrated and interoperating virtual observatory''}.

The operational VO itself is comprised of a distributed metadata registry, along
with data discovery and data access services deployed at each participating
institutes, which together enable the system to present a uniform mechanism for
discovering and accessing data published in the virtual observatory,
irrespective of where it is physically located.

\subsection{The registry}

The VO Registry provides the first layer of data discovery available in the
virtual observatory. The individual registry services deployed at participating
institutes work together to provide a shared repository for describing datasets,
data access services and data processing services in a standard way.

The IVOA Registry Interfaces standard
\endnote{http://www.ivoa.net/documents/RegistryInterface/}
defines 
\textit{``the interfaces that support interactions between applications and
registries as well as between the registries themselves''}.

The high level structure of the registry content is defined by a
set of IVOA standards.
\begin{itemize}
  \item IVOA Identifiers
  \endnote{http://www.ivoa.net/documents/latest/IDs.html}
  \item Resource Metadata
  \endnote{http://www.ivoa.net/Documents/latest/RM.html}
\end{itemize}

Combined with a set of lower level standards defining the detailed XML schemas
for resource metadata.
\begin{itemize}
  \item VOResource
  \endnote{http://www.ivoa.net/documents/latest/VOResource.html}
  \item VODataService
  \endnote{http://www.ivoa.net/documents/VODataService/} 
  \item Simple Data Access Services
  \endnote{http://www.ivoa.net/documents/SimpleDALRegExt/20131005/}
\end{itemize}

\subsection{Service registration}

VOSI \ldots and stuff \ldots

\subsection{Service metadata}

registry metadata queries \ldots and stuff \ldots

\subsection{Service footprint}

registry footprint queries \ldots and stuff \ldots

HEALPix Multi-Order Coverage Map (MOC)
\endnote{http://www.ivoa.net/documents/MOC/}

\subsection{Data access services}

The VO DataAccess services can be categorised as two types of services.

A set of type specific data discovery servces which are designed to provide
simple service interfaces for discovering and acessing data of a specific type.

\begin{itemize}
  \item Simple Cone Search (SCS)
  \endnote{http://www.ivoa.net/documents/latest/ConeSearch.html}
  \item Simple Image Access (SIA)
  \endnote{http://www.ivoa.net/documents/SIA/}
  \item Simple Spectral Access (SSA)
  \endnote{http://www.ivoa.net/documents/SSA/}
  \item Simple Line Access (SLA)
  \endnote{http://www.ivoa.net/documents/SLAP/}
\end{itemize}

A tabular data access services for querying tabluar data using a common query
derrived from SQL.

\begin{itemize}
  \item Table Access Protocol (TAP)
  \endnote{http://www.ivoa.net/Documents/TAP/}
  \item Astronomy Data Query Language (ADQL)
  \endnote{http://www.ivoa.net/Documents/latest/ADQL.html}
\end{itemize}

\subsubsection{Simple Image Access}

The Simple Image Access (SIA) protocol provides 
\begin{quote}
parameter based discovery of images and datacubes, querying the service(s) with
a few well known kinds of queries that cover greater than 95\% of use, and
getting back easily parsed summary metadata about each available data product
\end{quote}

The Simple Image Access (SIA) data discovery service provides support for
the following use cases:

\begin{itemize}
  \item find data that includes specified coordinates (e.g. for some object) 
  \item find data in the circle with coordinate centre and radius 
  \item find data in a range of longitude and latitude 
  \item find data within a specified simple  polygon (one region, no holes, less
  than half the sphere)
  \item find data containing a specified energy (e.g. wavelength) or in a
  specified range of energy values
  \item find data obtained at a specified time (e.g. including a time instant)
  or during a specified range of times
  \item find data obtained with specified polarization (Stokes) states 
  \item find data within a specified range of spatial resolution 
  \item find data within a specified range of field-of-view 
  \item find data within a range of exposure (integration) time 
\end{itemize}

The response from a successful SIA data discovery query is a VOTable containing
instances of the ObsCore \endnote{http://www.ivoa.net/documents/ObsCore/} data
model.

Each row in the results corresponds to a data product that matches the search
criteria and includes details of how to access the data products or how
to request additional matadata.

\subsubsection{Simple Spectral Access}

The Simple Spectral Access (SSA) protocol is similar to the Simple Image Access (SIA) protocol.

The primary differences are the type of data searched for, and the set of query
parameters.

\begin{quote}
\ldots discover and access one dimensional spectra
\ldots based on a general data model capable of describing most tabular
spectrophotometric data, including time series and spectral energy distributions
(SEDs) as well as 1-D spectra
\end{quote}

\subsubsection{Simple Line Access}

The Simple Line Access (SLA) protocol is similar to the Simple Image Access
(SIA) protocol.

The primary differences are the type of data searched for, and the set of query
parameters.

\begin{quote}
\ldots retrieving spectral lines coming from various Spectral Line Data
Collections
\ldots either observed or theoretical and will be typically used to identify
emission or absorption features in astronomical spectra.
\ldots makes use of the Simple Spectral Line Data Model
(SSLDM) \endnote{http://www.ivoa.net/documents/SSLDM/} to characterize spectral
lines through the use of uTypes
\endnote{http://www.ivoa.net/documents/Notes/UTypesUsage/index.html}
\end{quote}
  
\subsubsection{Table Access Protocol}

Table Access Protocol (TAP) is a generic protocol for accessing general table
data, including astronomical catalogs as well as general database tables, with
support for both synchronous and asynchronous queries.

Special support is provided for spatially indexed queries using the spatial
extensions in ADQL.

A multi-position query capability permits queries against an arbitrarily large
list of astronomical targets, providing a simple spatial cross-matching
capability.

Deploying the same standard interface and query language across multiple sites
means that cross-matching queries are possible by orchestrating a distributed
query across multiple TAP services.

\subsection{Data discovery example}

The combination of data discoverry tools provided by the registry and the
individual data access services working together enable the kind of \textit{'whole
sky'} data discovery queries that the virtual observatory is designed to
provide.

As an example of how this works, the following sections will describe the IVOA
standards and services involved in processing a data discovery query for a
particular type of data, e.g. images, covering a particular region of the sky,
in a particular wavelength e.g. infrared, visible light, radio or xray.

\subsubsection{Service discovery}

The first step in processing our \textit{'whole sky'} search is to identify the
set of services that contain the type of data we are looking for.

The example search is looking for images, so the first step is to query the
registry to find services that offer a \texttt{ivo://ivoa.net/std/SIA}
(SIA) capability.

The registry query can be refined by selecting services that contain data in the
\texttt{optical} waveband.

The MOC coverage map for each service can be used to further filter the list of
services to identify services that contain data in a particular region of the sky.

\subsubsection{Data discovery}

The next stage of the process is to send a SIA query to each of the SIA services
returned from the service discovery stage.

\noindent
SIA query can specify parameters for a particular wavelength and  a particular
region of the sky :
\begin{itemize}
  \item \texttt{POS}  The positional region
  \item \texttt{BAND} The the energy interval
\end{itemize}

The full list of SIA search parameters is given in appendix A.

Each SIA query would return a VOTable each row of which is an instance of the
ObsCore \endnote{http://www.ivoa.net/documents/ObsCore/} data model.

The ObsCore data model specifies all the VOTable field names, utypes, UCDs, and
units to use in the response.

The final step is for the client software to aggregate the results from the
individual SIA services and display them to the user.

The user selects which data products they are interested in, and the client
software uses the information from the SIA results to download the individual
data products and launch the appropriate display tools.









\appendix
\section{SIA search parameters}
\noindent
The SIA search parameters include
\begin{itemize}
  \item \texttt{POS}  The positional region
  \item \texttt{BAND} The the energy interval
  \item \texttt{TIME} The the time interval
  \item \texttt{POL}  The the polarization state
  \item \texttt{FOV}  The the field of view
  \item \texttt{SPATRES} The the spatial resolution
  \item \texttt{EXPTIME} The the exposure time
  \item \texttt{COLLECTION} The name of the data collection that contains the data
  \item \texttt{FACILITY} The name of the facility where the data was acquired
  \item \texttt{INSTRUMENT} The name of the instrument with which the data was acquired
  \item \texttt{DPTYPE} The data type from the ObsCore
  \endnote{http://www.ivoa.net/documents/ObsCore/} data model
  \item \texttt{CALIB} The calibration level
  \item \texttt{TARGET} The target name from the ObsCore
  \endnote{http://www.ivoa.net/documents/ObsCore/} data model
  \item \texttt{TIMERES} The temporal resolution
  \item \texttt{SPECRP} The spectral resolving power
  \item \texttt{FORMAT} The data format
\end{itemize}



\theendnotes

\end{document}
