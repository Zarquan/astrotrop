%%This is a very basic article template.
%%There is just one section and two subsections.
\documentclass{article}

\usepackage{endnotes}
	
\usepackage[utf8]{inputenc}
\usepackage[english]{babel}
 
\setlength{\parindent}{4em}
\setlength{\parskip}{1em}

\begin{document}

\section{Introduction}

Some text \ldots

\section{The IVOA Virtual Observatory}

The International Virtual Observatory Alliance (IVOA)
\endnote{http://www.ivoa.net/}
was formed in June 2002 with a mission to
\begin{quote}
"facilitate the international coordination and collaboration
necessary for the development and deployment of the tools, systems and
organizational structures necessary to enable the international utilization of
astronomical archives as an integrated and interoperating virtual observatory."
\end{quote}

The work of the IVOA focuses on the development of standards, providing a forum
for members to debate and agrees the technical standards that are needed to make
the VO possible.

The Virtual Observatory (VO) is the realization of the IVOA vision of an
integrated and interoperating virtual observatory.

The operational VO itself is comprised of a global shared metadata registry,
along with individual data discovery and data access services deployed at each
of the participating institutes, which work together to present a uniform
mechanism for discovering and accessing data, irrespective of where it is
physically located.

The VO architecture and data discovery process is very similar to the
interconnected metadata collections approach described in a paper on 'The new
bioinformatics: integrating ecological data from the gene to the biosphere'
(Jones et al. 2006).
\endnote{The new bioinformatics: integrating ecological data from the gene to the biosphere (Jones et al. 2006)}
\endnote{doi:10.1146/annurev.ecolsys.37.091305.110031}
\endnote{http://www.annualreviews.org/doi/abs/10.1146/annurev.ecolsys.37.091305.110031}

\begin{quote}
"An alternative, more robust approach to the highly structured, vertically integrated
data warehouse is a more loosely structured collection of project-specific data sets
accompanied by structured metadata about each of the data sets."
\end{quote}

\begin{quote}
"Each of the data sets is stored in a manner that is opaque to the data system in
that the data themselves cannot be directly queried; rather, the structured
metadata describing the data is queried in order to locate data sets of
interest. After data sets of interest are located, more detailed information
(such as the detailed data model that specifies, e.g., the definitions of the
variables) can be extracted from the metadata and used to load, query, and
manipulate individual data sets."
\end{quote}

\subsection{Example use case}

A useful way to illustrate how the VO data discovery process works is to look at
an example task such as selecting images covering a particular region of the
sky, in a particular wavelength e.g. infrared, visible light, radio or xray,

\subsubsection{Service discovery}

The first step of the process is to identify the services that provide access to
the type of data we are looking for by querying the VO Registry.

The VO Registry is comprised of a number of small local registries, typically
hosted at the participating institute level, working in cooperation with a set
of higer level global registries typically hosted by a few key institutes that
aggregate the data from the smaller registries to create a global searchable
index of metadata about all the services and datasets in the VO.

When a new service is deployed, part of the deployment process involves
registering the service with the local registry. The local registry is then
responsible for collecting and storing the metadata that describes both the
service itself and the datasets that it provides access to.

Once the metadata is registered in a local registry, it is automatically
propagated up to the next level and replicated between the global registries.

This means that a client can access all of the available metadata for all of the
services and datasets in the VO by querying any of the global registries.

The first step in fulfilling our example use case is to identify services that
contain the type of data we are looking for, in this case images, by querying
the registry for services that support the Simple Image Access (SIA)
\endnote{http://www.ivoa.net/documents/SIA/}
capability.

In addition to the technical details of services and their capabilities the
regitry metadata also contains details about the content of datasets, including
details of the wavelength(s) measured, e.g. infrared, visible, radio or xray.

This allows us to refine our query to search for SIA services that contain
images in a specific waveband, e.g. \texttt{optical} or \texttt{infrared}

The registry query would return a table of data, each row of which contains
information about a SIA services that contains the type of data we are
interested in - images in a particular wavelength.

The VO is itself an evolving system, building on the existing work to add
additional levels of integration as new features are added to the IVOA
specifications.

A recent addition to the IVOA is the HEALPix Multi-Order Coverage Map (MOC)
\endnote{http://www.ivoa.net/documents/MOC/}
which will allow registry services to perform coarse grained region matches.

Once this is in place we should be able to further refine our query to filter
for SIA services that contained data in a particular region of the sky.

\subsubsection{Data discovery}

The next stage of the process is to query each of the SIA services in the list
to discover details about the individual images available from that service.

\noindent
A SIA query can specify parameters for a particular wavelength and a particular
region of the sky :
\begin{itemize}
  \item \texttt{POS}  The positional region
  \item \texttt{BAND} The the energy interval
\end{itemize}

Each SIA service would return a table of data, each row of which contains
metadata about an individual image. The details of the fields in the
image metadata are defined in the ObsCore
\endnote{http://www.ivoa.net/documents/ObsCore/}
data model.

As every one of the SIA services returns a standard response, it makes it easy
to combine them to produce a single list of all the images available in the VO
that match our search criteria.

The user can then select which data products they are interested in, and their
client software can use the metadata in the SIA results to access the individual
data products and display them in the appropriate tools.

\noindent
The two key components of this process are
\begin{itemize}
  \item A standard interface for the global registry that uses a standard set of
  attributes to describe datasets and services
  \item A standard interface for local data access services that uses a standard
  set of attributes to describe the available data products
\end{itemize}

The separation between the initial service discovery query at the global
level followed by individual data discovery queries at the local level
are very similar to the stages described in (Jones et al. 2006) of first
querying the metadata to establish the data location followed by a more detailed
individual queries to establish what the data is and how to access it.

\subsection{Carbon density comparison}

We can compare the VO data discovery process for astronomy data with an example
use case based on a recent study 'Markedly divergent estimates of Amazon forest
carbon density from ground plots and satellites' (Mitchard et al. 2014)
\endnote{doi:10.1111/geb.12168}
\endnote{http://onlinelibrary.wiley.com/doi/10.1111/geb.12168/abstract}
comparing remote sensing data from satelites with ground plot data collected in
the field.

The study compares two sets of remote sensing data, from
the NASA Jet Propulsion Laboratory
(Saatchi et al., 2011)
\endnote{doi:10.1073/pnas.1019576108}
\endnote{http://www.pnas.org/content/108/24/9899}
and the Woods Hole Research Center
(Baccini et al., 2012)
\endnote{doi:10.1038/nclimate1354}
\endnote{http://www.nature.com/nclimate/journal/v2/n3/full/nclimate1354.html}
with four sets of ground plot data from
Red Amazónica de Inventarios Forestales (RAINFOR) (Malhi et al. 2002)
\endnote{http://www.rainfor.org/}
the Amazon Tree Diversity Network (ATDN) (ter Steege et al., 2003)
\endnote{http://web.science.uu.nl/Amazon/ATDN/}
the Tropical Ecology Assessment and Monitoring (TEAM)
\endnote{http://www.teamnetwork.org/}
network and the
Brazilian Program for Biodiversity Research
(PPBio).
\endnote{doi:10.7809/b-e.00083}
\endnote{http://www.biodiversity-plants.de/biodivers_ecol/article_meta.php?DOI=10.7809/b-e.00083}

\subsubsection{Satelite data}






\subsubsection{Ground plot data}

In order to calculate a single above ground biomass (AGB) dataset,
the ground plot data were brought together in the ForestPlots.Net
(Lopez-Gonzalez et al. 2009, 2011)
\endnote{http://www.forestplots.net/}
\endnote{http://onlinelibrary.wiley.com/doi/10.1111/j.1654-1103.2011.01312.x/abstract}
\endnote{doi:10.1111/j.1654-1103.2011.01312.x}
database.

ForestPlots.Net is a website and database designed to provide a repository for
long-term intact tropical forest inventory plots, where trees within an area are
individually identified, measured and tracked through time.

In addition to the raw measurements of tree diameter, the ForestPlots.Net database
stores a comprehensive set of metadata including taxonomic information about
the individual trees and detailed metadata about the plots.

Of the three sets of ground plot data, the data from RAINFOR and ATDN were
already available in the ForestPlots.Net database. The plot data from the TEAM
and PPBio projects were manually downloaded and imported into the
ForestPlots.Net database.

The principal AGB dataset was calculated using a tropical forest model described in Chave et al. (2005),
\endnote{Tree allometry and improved estimation of carbon stocks and balance in tropical forests}
\endnote{http://link.springer.com/article/10.1007/s00442-005-0100-x}
\endnote{doi:10.1007/s00442-005-0100-x}
using one of the built-in SQL queries provided by the ForestPlots.Net database
system.

The resulting data set was itself stored in the ForestPlots.Net database as
a new dataset available for download as part of the source material for the paper.






\subsection{Diverse metadata formats}

Within just the datasets used by our example use case, we can see a variety of
different database systems storing different types of metadata in a varierty of
different structures and formats.

\subsubsection{GIVD}

ForestPlots.Net is itself part of a hierarchy of databases containing metadata
about databases.

The Global Index of Vegetation-Plot Databases (GIVD)
\endnote{http://www.givd.info/}
is a database of metadata describing databases of vegetation plot data
from around the world.

\noindent
ForestPlots.Net is described in the GIVD database
\texttt{[GIVD:00-00-001]}
\endnote{http://www.givd.info/ID/00-00-001}
as is the Brazilian Program for Biodiversity Research (PPBio) information system
\texttt{[GIVD:SA-BR-001]}	
\endnote{http://www.givd.info/ID/SA-BR-001}
and the data from the 
Tropical Ecology Assessment and Monitoring (TEAM) network
\texttt{[GIVD:00-00-002]}.
\endnote{http://www.givd.info/ID/00-00-002}


The GIVD database authors describe the project in a 2011 paper,
"The Global Index of Vegetation-Plot Databases (GIVD): a new resource for vegetation science"
(Dengler et al. 2011)
\endnote{doi:10.1111/j.1654-1103.2011.01265.x}
\endnote{http://onlinelibrary.wiley.com/doi/10.1111/j.1654-1103.2011.01265.x/abstract}
and suggest some future applications, including the idea of combining
different types of data from different, distributed, databases.

\begin{quote}
"Our longer-term vision is to develop GIVD in ways similar to Metacat (Jones et
al. 2006), so that, ultimately, users who query GIVD will not only receive
information on which databases contain data suitable for the intended analyses,
but they will also discover other data from distributed databases, with GIVD
acting as the central node."
\end{quote}

\begin{quote}
"By coupling species specific trait characteristics (e.g. mean plant height,
specific leaf area, growth form) found in trait databases, such as LEDA (Kleyer
et al. 2008) or TRY (http://www.trydb.org), with plot-based distribution
information on those species, GIVD could support further refinement of DGVMs."
\end{quote}

Their idea of many different databases working together to create a larger
system has a lot of similarities with the distributed architectrure of data
discovery and data access services working together to create the virtual
observatory.

\subsubsection{METACAT}

From paper describing PPBio
\endnote{http://www.biodiversity-plants.de/biodivers_ecol/article_meta.php?DOI=10.7809/b-e.00083}

\begin{quote}
To facilitate data searches, all the metadata were converted to XML, and the
PPBio has installed a METACAT server to integrate with the Knowledge Network for
Biocomplexity (KNB), a network which aims to assist ecological and environmental
research.
\end{quote}

METACAT
\endnote{http://knb.ecoinformatics.org/software/metacat}
\endnote{https://knb.ecoinformatics.org/knb/docs/intro.html}

\begin{quote}
Metacat is a repository for data and metadata (documentation about data) that
helps scientists find, understand and effectively use data sets they manage or
that have been created by others. Thousands of data sets are currently
documented in a standardized way and stored in Metacat systems, providing the
scientific community with a broad range of science data that–because the data
are well and consistently described–can be easily searched, compared, merged, or
used in other ways.
\end{quote}
\begin{quote}
Metacat is a Java servlet application that runs on Linux, Mac OS, and Windows
platforms in conjunction with a database, such as PostgreSQL (or Oracle), and a
Web server.
\end{quote}
\begin{quote}
The Metacat application stores data in an XML format using Ecological Metadata
Language (EML) or other metadata standards such as ISO 19139 or the FGDC
Biological Data Profile.
\end{quote}

\subsubsection{DataONE}

Metacat features include providing a Data Observation Network for Earth (DataONE)
\endnote{https://www.dataone.org/}
node service.
 
\begin{quote}
The DataONE project is a collaboration among scientists, technologists,
librarians, and social scientists to build a robust, interoperable, and
sustainable system for preserving and accessing Earth observational data at
national and global scales. Supported by the U.S. National Science Foundation,
DataONE partners focus on technological, financial, and organizational
sustainability approaches to building a distributed network of data repositories
that are fully interoperable, even when those repositories use divergent
underlying software and support different data and metadata content standards.
\end{quote}
\begin{quote}
DataONE defines a common web-service service programming interface that allows
the main software components of the DataONE system to seamlessly communicate.
\end{quote}








\subsection{VO services}

\subsection{The registry}

The VO Registry provides the first layer of data discovery available in the
virtual observatory. The individual registry services deployed at participating
institutes work together to provide a shared repository for describing datasets,
data access services and data processing services in a standard way.

The IVOA Registry Interfaces standard
\endnote{http://www.ivoa.net/documents/RegistryInterface/}
defines the web service interfaces that support interactions between
applications and registries as well as between the registries themselves.

The high level structure of the registry content is defined by a
set of IVOA standards, including a standard format for IVOA Identifiers
\endnote{http://www.ivoa.net/documents/latest/IDs.html}
and the basic Resource Metadata
\endnote{http://www.ivoa.net/Documents/latest/RM.html}

The details of the registry metadata are covered by a set of technical
standards defining the detailed XML schemas for resource metadata.
\begin{itemize}
  \item VOResource
  \endnote{http://www.ivoa.net/documents/latest/VOResource.html}
  \item VODataService
  \endnote{http://www.ivoa.net/documents/VODataService/} 
  \item Simple Data Access Services
  \endnote{http://www.ivoa.net/documents/SimpleDALRegExt/20131005/}
\end{itemize}

\subsection{Service registration}

VOSI \ldots and stuff \ldots

\subsection{Service metadata}

registry metadata queries \ldots and stuff \ldots

\subsection{Service footprint}

registry footprint queries \ldots and stuff \ldots

HEALPix Multi-Order Coverage Map (MOC)
\endnote{http://www.ivoa.net/documents/MOC/}

\subsection{Data access services}

The VO DataAccess services can be categorised as two types of services.

A set of type specific data discovery servces which are designed to provide
simple service interfaces for discovering and acessing data of a specific type.

\begin{itemize}
  \item Simple Cone Search (SCS)
  \endnote{http://www.ivoa.net/documents/latest/ConeSearch.html}
  \item Simple Image Access (SIA)
  \endnote{http://www.ivoa.net/documents/SIA/}
  \item Simple Spectral Access (SSA)
  \endnote{http://www.ivoa.net/documents/SSA/}
  \item Simple Line Access (SLA)
  \endnote{http://www.ivoa.net/documents/SLAP/}
\end{itemize}

A tabular data access services for querying tabluar data using a common query
derrived from SQL.

\begin{itemize}
  \item Table Access Protocol (TAP)
  \endnote{http://www.ivoa.net/Documents/TAP/}
  \item Astronomy Data Query Language (ADQL)
  \endnote{http://www.ivoa.net/Documents/latest/ADQL.html}
\end{itemize}

\subsubsection{Simple Image Access}

The Simple Image Access (SIA) protocol provides 
\begin{quote}
parameter based discovery of images and datacubes, querying the service(s) with
a few well known kinds of queries that cover greater than 95\% of use, and
getting back easily parsed summary metadata about each available data product
\end{quote}

The Simple Image Access (SIA) data discovery service provides support for
the following use cases:

\begin{itemize}
  \item find data that includes specified coordinates (e.g. for some object) 
  \item find data in the circle with coordinate centre and radius 
  \item find data in a range of longitude and latitude 
  \item find data within a specified simple  polygon (one region, no holes, less
  than half the sphere)
  \item find data containing a specified energy (e.g. wavelength) or in a
  specified range of energy values
  \item find data obtained at a specified time (e.g. including a time instant)
  or during a specified range of times
  \item find data obtained with specified polarization (Stokes) states 
  \item find data within a specified range of spatial resolution 
  \item find data within a specified range of field-of-view 
  \item find data within a range of exposure (integration) time 
\end{itemize}

The response from a successful SIA data discovery query is a VOTable containing
instances of the ObsCore \endnote{http://www.ivoa.net/documents/ObsCore/} data
model.

Each row in the results corresponds to a data product that matches the search
criteria and includes details of how to access the data products or how
to request additional matadata.

\subsubsection{Simple Spectral Access}

The Simple Spectral Access (SSA) protocol is similar to the Simple Image Access (SIA) protocol.

The primary differences are the type of data searched for, and the set of query
parameters.

\begin{quote}
\ldots discover and access one dimensional spectra
\ldots based on a general data model capable of describing most tabular
spectrophotometric data, including time series and spectral energy distributions
(SEDs) as well as 1-D spectra
\end{quote}

\subsubsection{Simple Line Access}

The Simple Line Access (SLA) protocol is similar to the Simple Image Access
(SIA) protocol.

The primary differences are the type of data searched for, and the set of query
parameters.

\begin{quote}
\ldots retrieving spectral lines coming from various Spectral Line Data
Collections
\ldots either observed or theoretical and will be typically used to identify
emission or absorption features in astronomical spectra.
\ldots makes use of the Simple Spectral Line Data Model
(SSLDM) \endnote{http://www.ivoa.net/documents/SSLDM/} to characterize spectral
lines through the use of uTypes
\endnote{http://www.ivoa.net/documents/Notes/UTypesUsage/index.html}
\end{quote}
  
\subsubsection{Table Access Protocol}

Table Access Protocol (TAP) is a generic protocol for accessing general table
data, including astronomical catalogs as well as general database tables, with
support for both synchronous and asynchronous queries.

Special support is provided for spatially indexed queries using the spatial
extensions in ADQL.

A multi-position query capability permits queries against an arbitrarily large
list of astronomical targets, providing a simple spatial cross-matching
capability.

Deploying the same standard interface and query language across multiple sites
means that cross-matching queries are possible by orchestrating a distributed
query across multiple TAP services.

 
















\appendix
\section{SIA search parameters}
\noindent
The SIA search parameters include
\begin{itemize}
  \item \texttt{POS}  The positional region
  \item \texttt{BAND} The the energy interval
  \item \texttt{TIME} The the time interval
  \item \texttt{POL}  The the polarization state
  \item \texttt{FOV}  The the field of view
  \item \texttt{SPATRES} The the spatial resolution
  \item \texttt{EXPTIME} The the exposure time
  \item \texttt{COLLECTION} The name of the data collection that contains the data
  \item \texttt{FACILITY} The name of the facility where the data was acquired
  \item \texttt{INSTRUMENT} The name of the instrument with which the data was acquired
  \item \texttt{DPTYPE} The data type from the ObsCore
  \endnote{http://www.ivoa.net/documents/ObsCore/} data model
  \item \texttt{CALIB} The calibration level
  \item \texttt{TARGET} The target name from the ObsCore
  \endnote{http://www.ivoa.net/documents/ObsCore/} data model
  \item \texttt{TIMERES} The temporal resolution
  \item \texttt{SPECRP} The spectral resolving power
  \item \texttt{FORMAT} The data format
\end{itemize}



\theendnotes

\end{document}
